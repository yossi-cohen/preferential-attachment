\documentclass[a4paper, 12pt]{report}
\usepackage{geometry}
\geometry{a4paper,hmargin={3cm,2.5cm},vmargin={2.5cm,2.5cm}}

\usepackage{geometry}
\geometry{
    a4paper,
    headheight=15pt,
    hmargin={3cm,2.5cm},
    vmargin={2.5cm,2.5cm}
}
 
\usepackage{amsfonts} % if you want blackboard bold symbols e.g. for real numbers
\usepackage{multicol}
\usepackage{float}
\usepackage{blindtext}
\usepackage{fancyhdr}
\pagestyle{fancy}
\usepackage{hyperref}
\usepackage{amsmath} % for 'bmatrix' environment
\usepackage{tikz}
\usetikzlibrary{calc}
\usepackage{eso-pic}
\usepackage{lipsum}
\usepackage{longtable}
\setlength{\parindent}{0em}
\setlength{\parskip}{1em}
\usepackage{biblatex} %Imports biblatex package
\addbibresource{main.bib} %Import the bibliography file
\usepackage{scrextend}
\usepackage{subfig}
\usepackage{graphicx} % for jpeg or pdf pictures
\usepackage{lmodern}
\usepackage{minted}
\usepackage{colortbl}
\usepackage[section]{placeins}

\hypersetup{
    colorlinks=true,
    linkcolor=blue,
    filecolor=magenta,      
    urlcolor=cyan,
}

%\lhead{}
%\rhead{Project Report - Preferential Attachment with machine learning}
%\lfoot{Open University of Israel - Dept. of Mathematics and Computer Science}

\begin{document}

%========================== begin of title page ================================ %
% The frontmatter environment for everything that comes with roman numbering
\newenvironment{frontmatter}{}{}
\begin{frontmatter}
%%%%%%%%%%%%%%%%%%%%%%%%%%%%%%%%%%%%%%%%%%%%%%%%%%%%%%%%%%%%%%%%%%%
\begin{titlepage}
%\AddToShipoutPictureBG
\begin{center}
\textup{\large \textbf{The Open University of Israel}}\\[0.5cm]
\textbf{\large Department of Mathematics and Computer Science}

%-------------------------------------- Figure -----------------------------------

\begin{center}
\begin{figure}[h]  %h means here other options t , b, p, etc.
\centering
\includegraphics[width=0.3\linewidth]{./logo}
\end{figure}
\end{center}

%---------------------------------------------------------------------------------

\textup{\large A PROJECT REPORT\\[0.4cm]ON}\\[0.4cm]
\begin{LARGE}
{\textbf {Preferential Attachment }}\\
{\textbf {using machine learning }}\\[1cm]
\end{LARGE}
\textit{SUBMITTED BY}\\[0.3cm]
\begin{large}
\textbf{YOSSI COHEN}\\[0.3cm]
\end{large}
\textit{THE OPEN UNIVERSITY}\\[1cm]
\textit{UNDER THE GUIDANCE OF}\\[0.3cm]
\begin{large}\textbf{PROF. YOSSI GIL}\\[0.3cm]\end{large}
\textit{THE TECHNION}\\[1cm]
\textit{(2020-2021)}
\vfill
\end{center}
\end{titlepage}

%========================== end of title page ================================ %

%------------------------------ ACKNOWLEDGEMENT --------------------------------
\begin{center}
{\Large{\bf{\textit{ACKNOWLEDGEMENT}}\\[2cm]}}
\end{center}

\par\textit{I would like to thank {\bf Prof. Yossi Gil} for encouraging me to go ahead and for his continuous guidance with the project and assistance in preparing this report.}

\par\textit{I would also like to thank all those, who have directly or indirectly helped me for the completion of the work during this project.}

%========================= table of contents ================================= %
\tableofcontents
\listoffigures
\listoftables

%%%%%%%%%%%%%%%%%%%%%%%%%%%%%%%%%%% Abstract %%%%%%%%%%%%%%%%%%%%%%%%%%%%%%%%%%%%
%\newpage
\begin{abstract}
Power laws appear widely in physics, biology, earth and planetary sciences, economics and finance, computer science, demography and social sciences. For instance, the distributions of the sizes of cities, earthquakes, forest fires, solar flares, moon craters and people’s personal fortunes all appear to follow power laws.

In his paper "Power laws, Pareto distributions and Zipf’s law", Newman review some of the empirical evidence for the existence of power-law forms and the theories proposed to explain them.

{\bf In this project} we show, that a key property of such \textit{power law} distribution, concretely  \textit{Yule-Simon distribution}, can be in fact estimated with a reasonable accuracy using machine learning methods.

\vspace{0.3cm}

\textbf{Keywords - }\it{\textbf{Preferential Attachment, Machine Learning.}}
\end{abstract}
%================================================ %
% The frontmatter environment for everything that comes with roman numbering %
\end{frontmatter}
%%%%%%%%%%%%%%%%%%%%%%%%%%%%%%%%%%%%%%%%%%%%%%%%%%%%%%%%%%%%%%%%%%%%%%%%%%%%%%%%%%%%%%%%%%%%%%%%

\newpage
\pagenumbering{arabic}
%%%%%%%%%%%%%%%%%%%%%%%%%%%%%%%%%%%% MAIN TEXT STARTS HERE %%%%%%%%%%%%%%%%%%%%%%%%%%%%%%%%%%%%%
\chapter{INTRODUCTION}

\section{Background}
In his paper "Power laws, Pareto distributions and Zipf’s law" \cite{newman}, Newman review some of the empirical evidence for the existence of power-law forms and the theories proposed to explain them. This section is mostly a short summary from this paper.

Many of the things that we measure have a \textit{typical value} around which individual measurements are centred. A simple example would be the heights of human beings. Most adult human beings (in the U.S) are about 180 cm tall. There is some variation around this figure, notably depending on sex, but we never see people who are 10 cm tall, or 500 cm. Other examples would be e.g., shoe sizes and speed of cars on the motorway.

\begin{figure}[!htb]
\centering
\includegraphics[width=0.7\linewidth]{./heights-of-males}
\caption{Newman-2004: distributions centered around a typical value}
\label{fig:heights-of-males}
\end{figure}

However, not all things we measure are peaked around a typical value. Some vary over an enormous dynamic range, sometimes many orders of magnitude. 

When the probability of measuring a particular value of some quantity varies inversely as
a power of that value, the quantity is said to follow a \textit{power law}, also known as \textit{Zipf’s law} or the \textit{Pareto distribution}. 

\newpage
Power laws appear widely in physics, biology, earth and planetary sciences, economics and finance, computer science, demography and social sciences. For instance, the distributions of the sizes of cities, earthquakes, forest fires, solar flares, moon craters and people’s personal fortunes all appear to follow power laws.

\begin{figure}[!htb]
\centering
\includegraphics[width=0.7\linewidth]{./population-of-cities}
\caption{Newman-2004: population of cities}
\label{fig:populations-of-cities}
\end{figure}

Figure \ref{fig:populations-of-cities} (left) shows the histogram of city sizes. On the same figure (right), the histogram is shown in a log-log scale. The histogram, plotted in this fashion follow quite closely a straight line (this observation is attributed to Zipf).

Let $p(x)dx$ be the fraction of cities with population between $x$ and $x+dx$. If the histogram is a straight line on log-log scales, then $ln p(x) = - \alpha ln(x) + c$, where $\alpha$ and $c$ are constants. Taking
the exponential of both sides, we get:

\begin{equation}
\label{eq:powerlaw}
p(x) = Cx^{- \alpha}
\end{equation}
with $C = e^c$.

Distributions of the form \ref{eq:powerlaw} are said to follow a power law. The constant $\alpha$ is called the exponent of the power law. (The constant $C$ is mostly uninteresting; once $\alpha$ is fixed, it is determined by the requirement that the distribution $p(x)$ sum to 1).

\newpage
\subsection{Measuring power laws}
The standard strategy for identifying power-law behaviour is to plot a histogram on a logarithmic scale where it appears as a straight line. Figure \ref{fig:histogram-1-million} shows a fake data set generated from 1 million random numbers with a power law distribution $p(x) = Cx^{- \alpha}$ with $\alpha=-2.5$

Panel (a) of the figure shows a normal histogram of the numbers, produced by binning them into bins of equal size $(0.1)$.

Plotting the histogram on logarithmic scales reveals the straight line behavior as can be seen in Panel (b). However, the right-hand end of the distribution is noisy because of sampling errors. In this region, each bin only has a few samples in it, if any. So the fractional fluctuations in the bin counts are large and this appears as a noisy curve on the plot.

In panel (c), the width of the bins in the histogram is plotted with \textit{logarithmic binning}, meaning the width of each bin is a multiple of the previous one $(0.1,0.2,0.4,...)$. The sample count is than normalized by dividing the number of samples in a bin of width $\Delta x$ by $\Delta x$. This means the bins in the tail of the distribution get more samples than they would if bin sizes were fixed, and this reduces the statistical errors in the tail.

\begin{figure}[!htb]
\centering
\includegraphics[width=0.7\linewidth]{./histogram-1-million}
\caption[Newman-2004: histograms of 1 million random numbers]{Newman-2004: histograms of 1 million random numbers with power law distribution}
\label{fig:histogram-1-million}
\end{figure}

Even with logarithmic binning there is still some noise in the tail, although it is sharply decreased. As long as $\alpha > 1$, the number of samples per bin goes down as $k$ increases and the bins in the tail will have more statistical noise than those that precede them. As shown in \cite{newman}, most power-law distributions occurring in nature have $2 \leq \alpha \leq 3$, so noisy tails are the norm.

Another method of plotting the data shown in panel (d) is to calculate a \textit{cumulative distribution function}, in which we make a plot of the probability $P(x)$ that $x$ has a value greater than or equal to $x$:

\begin{equation}
\label{eq:cumsum}
p(x) = \int_{x}^{\infty} p(x')\, dx'
\end{equation}

The \textit{cumulative distribution} also follows power law, but with an exponent of $\alpha-1 = 1.5$. Thus, we get again a straight line but with a shallower slope.

\section{Motivation}
We know the value of the exponent $\alpha$ for the artificial data set since it was generated deliberately to have a particular value, but in practical situations we would often like to \textbf{estimate $\alpha$ from observed data}.

As stated in \cite{newman}, one way to do this would be to fit the slope of the line in plots like figures \ref{fig:histogram-1-million} (b), (c) or (d), and this is the most commonly used method.
Unfortunately, it is known to introduce systematic biases into the value of the exponent [20], so it should not be relied upon. For example, a least-squares fit of a straight line to
figure 3 (b) gives a = 2.26 +0.02, which is clearly incompatible with the known value of a= 2.5 from which the data were generated.

\textbf{In this project}, we develop an alternative and reliable method for extracting the exponent with an acceptable error using \textit{Deep Learning} methods. Than, this preliminary result will motivate further investigation on harnessing such methods for identifying power law distributions and maybe classifying distributions as such.

\pagebreak
\section{Preferential Attachment}
\textit{Preferential Attachment} is a another model which has properties of the \textit{power law} distribution. It is best explaining the \textit{Preferential Attachment} model by an example (e.g: on a node network).
\begin{enumerate}
\small
    \item We start with two nodes connected by an edge.
    \item At each time step, add a new node with an edge connecting it to a single existing node. 
    \item Choose the node to connect to at random with probability proportional to the node`s current degree.
    \item The probability of connecting to a node $u$ of degree $k_u$ is $\frac{k_u}{\sum_j{k_j}}$
\end{enumerate}

\section{Yule process}
\par \textit{Preferential Attachment} may be realized using the \textit{Yule process}  \cite{yule-simon-distribution}.

\par An example \textit{Yule process} is implemented by following procedure:
\par Distribute $N >> 1$ balls among urns iteratively using \textit{preferential attachment} policy.

At each step, add a urn with initial $k_0$ balls and distribute $m$ balls among urns with probability proportional to the number of balls in each urn.\\
Run as many iterations as necessary to distribute all $N$ balls.

\par Parameters:
\small
\begin{addmargin}[1cm]{1cm}% 1cm left, 1cm right
    $N: N >> 1$, the total number of balls we wish to distribute.\\
    $m: 1 \leq m \leq 10$, the number of balls distributed at each iteration.\\
    $k_0: 1 \leq k_0 \leq 10$, the initial number of balls for a newly created urn.
\end{addmargin}

{\fontfamily{qcr}\selectfont
As an example: N = 10000; k0 = 10; m = 3
}

\begin{figure}[ht]
\centering
    % Requires \usepackage{graphicx}
    \subfloat[]{
        \includegraphics[width=0.25\textwidth]{yule-histogram}
        \label{fig:yule-process-histograms-1a}
    }
    \hspace{0.5cm}
    \subfloat[]{
        \includegraphics[width=.25\textwidth]{yule-log}
        \label{fig:yule-process-histograms-1b}
    }
    \hspace{0.5cm}
    \subfloat[]{
        \includegraphics[width=.25\textwidth]{yule-log-cumsum}
        \label{fig:yule-process-histograms-1c}
    }
    \label{fig:yule-process-histograms}
    \caption[Yule process histograms]{
        $(a)$ A histogram of $N=10,000$ balls distributed to urns with \textit{Yule distribution}.
        $(b)$ The same histogram on logarithmic scale.
        $(c)$ A cumulative histogram of the same data.
    }
\end{figure}

\chapter{PROJECT IMPLEMENTATION}

\section{Problem Statement}
\par Given a collection of generated samples drawn from the \textit{Yule-Simon} distribution with a known exponents $\alpha_1,...,\alpha_n$, the \textbf{task} of the learning model is to \textbf{predict $\alpha$} for a single sample of length $\geq m$ within a specified bounded error:
$$\mid prediced(\alpha) - \alpha \mid \; \leq \; 0.01$$

\par A simulator will be written to generate data for learning and testing purposes so there is no limit on the available dataset size.

\section{Software requirements}
The project was developed with \textit{Python 3.8} under \textit{Jupyter notebook}.

I advise on working with \textit{Python Virtual Environment}.

The \textit{Keras} framework was used for developing the machine learning models.

Source code can be found \href{https://github.com/yossi-cohen/preferential-attachment}{here}.

\textit{numpy/scipy} and other required python packages are listed in \textit{requirements.txt}.

\section{Dataset}
\label{dataset}

\par The datasets used in the project were generated using the \textbf{\textit{yulesimon}} distribution from \textbf{\textit{scipy.stat}} package.

\par As an example, the following will generate $1000$ random numbers for a given $alpha=\alpha$
\begin{equation}
\label{eq:yulesimon.rvs}
r = yulesimon.rvs(alpha=\alpha, size=1000)
\end{equation}
The $generate\_data()$ function uses \ref{eq:yulesimon.rvs} to draw $N$ random samples for each $\alpha$ in the range: $[min\_alpha, max\_alpha]$ with $S$ $yulesimon.rvs$ calls for each $\alpha$, yielding the following dataset:\\
(Concatenated as a single matrix $YS = [(M * S), N]$)
\begin{itemize}
\small
  \item $N$ (columns): number of samples drawn for a given $\alpha$
  \item $M$: number of different $\alpha$ generated in the range: [$min\_\alpha$, $max\_\alpha$]
  \item $S$: number of samples per $\alpha$ (number of calls to $yulesimon.rvs$ with same $\alpha$)
\end{itemize}
\[
samples(\alpha_1) = \begin{bmatrix} 
    r^{(\alpha_1)}_{1,1} & r^{(\alpha_1)}_{1,2} & \dots & r^{(\alpha_1)}_{1,N} & \\
    r^{(\alpha_1)}_{2,1} & r^{(\alpha_1)}_{2,2} & \dots & r^{(\alpha_1)}_{2,N} & \\
    \vdots & \ddots & \vdots & \\
    r^{(\alpha_1)}_{S,1} & r^{(\alpha_1)}_{S,2} & \dots & r^{(\alpha_1)}_{S,N} & \\
\end{bmatrix}
\]

\[
samples(\alpha_2) = \begin{bmatrix} 
    r^{(\alpha_2)}_{1,1} & r^{(\alpha_2)}_{1,2} & \dots & r^{(\alpha_2)}_{1,N} & \\
    r^{(\alpha_2)}_{2,1} & r^{(\alpha_2)}_{2,2} & \dots & r^{(\alpha_2)}_{2,N} & \\
    \vdots & \ddots & \vdots & \\
    r^{(\alpha_2)}_{S,1} & r^{(\alpha_2)}_{S,2} & \dots & r^{(\alpha_2)}_{S,N} & \\
\end{bmatrix}
\]
\[
\vdots
\]
\[
samples(\alpha_M) = \begin{bmatrix} 
    r^{(\alpha_M)}_{1,1} & r^{(\alpha_M)}_{1,2} & \dots & r^{(\alpha_M)}_{1,N} & \\
    r^{(\alpha_M)}_{2,1} & r^{(\alpha_M)}_{2,2} & \dots & r^{(\alpha_M)}_{2,N} & \\
    \vdots & \ddots & \vdots & \\
    r^{(\alpha_M)}_{S,1} & r^{(\alpha_M)}_{S,2} & \dots & r^{(\alpha_M)}_{S,N} & \\
\end{bmatrix}
\]

\par A histogram $H$ is then created for rows of $[YS]$ where: 
\begin{equation}
\label{eq:nbins}
\textbf {nbins} = max(YS)
\end{equation}
\[
{\textbf H} = \begin{bmatrix} 
    histogram(samples(\alpha_1), nbins) & \\
    histogram(samples(\alpha_2), nbins) & \\
    \vdots & \\
    histogram(samples(\alpha_S), nbins) & \\
\end{bmatrix}
\]

\par Finally, we apply $log_{10}$ on $H$ rows:
\begin{equation}
\label{eq:log_H}
    {\textbf X} = log_{10}(H)
\end{equation}

We set:
\begin{equation}
\label{eq:y}
{\textbf y} = \begin{bmatrix} 
    \alpha_1 & \\
    \alpha_2 & \\
    \vdots & \\
    \alpha_M & \\
\end{bmatrix}
\end{equation}

\begin{itemize}
  \item note: $y$ contains $S$ copies for each of $\alpha_1...\alpha_M$ for total length of $[S*M]$
\end{itemize}

The $generate\_data()$ function returns: $(X, y, nbins)$ as an input for the learning process.

\newpage
\par An example of a single row in $X = log(H)$ is shown in figure  \ref{fig:yule-simon-log-scale}

\begin{figure}[h]
\centering
\includegraphics[width=0.7\linewidth]{./dataset}
\caption[Yule-Simon log-scale]{$X = log(H)$ for $\alpha\in[2.0,3.0], N=2048$ \\(140 bins = $max(X)$)}
\label{fig:yule-simon-log-scale}
\end{figure}

\textit{Note}: We will see later that padding bins with zero value at the right end improves the $NN$ model accuracy (figure  \ref{fig:yule-simon-log-scale-zeros-padding})

\begin{figure}[h]
\centering
\includegraphics[width=0.7\linewidth]{./dataset-zeros}
\caption[Yule-Simon log-scale with zero padding]{Yule-Simon samples for $\alpha\in[2.0,3.0], N=2048$ \\(right padding zero bins)}
\label{fig:yule-simon-log-scale-zeros-padding}
\end{figure}

\pagebreak
\section{DNN}
\par As stated in \cite{newman}, In practical situations we would often like to estimate $\alpha$ from observed data. One way to do this would be to fit the slope of the line in plots like figures (\ref{fig:yule-process-histograms-1b}), (\ref{fig:yule-process-histograms-1c}) Unfortunately, it is known to introduce systematic biases into the value of the exponent $\alpha$, so it should not be relied
upon.

\par \textit{DNN (Deep Neural Network)}, not like simple line-fit models, can learn complex patterns. Thus, the first approach used in this project was a \textit{Regression DNN}.

\subsection{DNN Model training}
The \textit{DNN} model was implemented using \textit{Keras} under \textit{Jupyter Notebook}.

\par The neural network has an input layer which is fed with the dataset described in section \ref{dataset}. It has couple of fully connected hidden layers with \textit{relu} activation and an output layer with a \textit{linear} activation. All layers use batch normalization, \textit{Adam} optimizer, kernel and activation \textit{regularization}. The loss function used is \textit{MSE}.

\par We have made little attempts to optimize the parameters of the network, and
so improved performance may be obtained by exploring other values.

\par Input to the network has been splitted to \textit{TRAIN}, \textit{VALIDATION} and \textit{TEST} sets.

\par Model training used \textit{early stopping} and \textit{reduce learning-rate on plateau}.\\
Training used \textit{batch size} of $32$ and couple of hundred \textit{epochs}.

\textit{Learning Curves} (train/validation) have been used to identify and eliminate \textit{overfitting}.

\pagebreak
\subsection{Experiments}
In order to streamline experiments, a $trial()$ helper function has been defined.\\
$trial()$ runs several iterations as defined by the following arguments:

\begin{itemize}
  \item $N\_range$: a set $\{I\}$ where $\{N = 2^{i\in I}\}$ are the number of samples to draw from \textit{yulesimon} distribution for a given $\alpha$.
  \item $hstack\_zeros$: whether to right append zeros to the generated \textit{dataset}.
  \item $\{random\_states\}$: a set of random state values to run the trained model in each iteration.
\end{itemize}

The $trial()$ function calls $generate\_data(N)$ to produce growing size datasets. For each \textit{dataset} generated, $trial()$ creates a corresponding \textit{DNN} model. It then iterates over the given $\{random\_states\}$ set to train and test the model.

The $trial()$ function returns:
\begin{itemize}
\small
  \item $[N]$: the list of $N's$ used in each iteration.
  \item $[sqrt\_mse]$: the corresponding $sqrt(mse)$ for each iteration.
  \begin{itemize}
    \item $sqrt\_mse = sqrt(mean\_squared\_error(\alpha\_test, \alpha\_pred))$
  \end{itemize}
  \item $[avg\_abs\_errors]$: avg of absolute errors for each iteration.
  \begin{itemize}
    \item $avg\_abs\_errors = avg(abs(\alpha\_test - \alpha\_pred))$
  \end{itemize}
  \item $[std\_abs\_errors]$: \textsc{std} of the absolute error for each iteration.
  \begin{itemize}
    \item $std\_abs\_errors = std(avg\_abs\_errors)$
  \end{itemize}
\end{itemize}

\pagebreak
\subsection{DNN results}
\par Running \textit{DNN} on $100$ alphas with 100 samples per alpha yield the following:

\begin{table}[h!]
    \centering
    \begin{tabular}{||c c c||} 
        \hline
        $N$ & input-shape & $avg(sqrt(mse(\alpha)))$ \\ [0.5ex] 
        \hline\hline
        32 & (10000, 34) & 0.04326 \\ 
        \hline
        64 & (10000, 20) & 0.01430 \\
        \hline
        128 & (10000, 46) & 0.01480 \\
        \hline
        256 & (10000, 21) & 0.01041 \\
        \hline
        512 & (10000, 33) & 0.00908 \\ 
        \hline
        1024 & (10000, 63) & 0.00985 \\ 
        \hline
        2048 & (10000, 140) & 0.01002 \\ 
        \hline
    \end{tabular}
    \caption[$mse(\alpha)$ for $DNN$]{$DNN - avg(sqrt(mse(\alpha)))$}
    \label{table:dnn-sqrt-mse}
\end{table}

\textit{note}: for each $N$, we run several train/test cycles, each with different random-state and average the results:
\begin{minted}{python}
for rs in random_states:
    model = train(model, X_train, y_train, ..., random_state=rs)
    y_pred = model.predict(X_test)
    sqrt_mse = sqrt(mse(y_test, y_pred))
    avg_sqrt_mse += sqrt_mse
    ...
avg_sqrt_mse \= len(random_states)
\end{minted}

\subsection{DNN with zero padding}
\par Running \textit{DNN} on the same input as above with right padding zeros yield the following:
\begin{table}[h!]
    \centering
    \begin{tabular}{||c c c||} 
        \hline
        $N$ & input-shape & $avg(sqrt(mse(\alpha)))$ \\ [0.5ex] 
        \hline\hline
        32 & (10000, 44) & 0.04341 \\ 
        \hline
        64 & (10000, 30) & 0.01435 \\
        \hline
        128 & (10000, 56) & 0.01506 \\
        \hline
        256 & (10000, 31) & 0.01066 \\
        \hline
        512 & (10000, 43) & 0.00798 \\ 
        \hline
        1024 & (10000, 73) & 0.01143 \\ 
        \hline
        2048 & (10000, 150) & 0.00789 \\ 
        \hline
    \end{tabular}
    \caption{$mse(\alpha)$ for $DNN$ with zero padding}
    \label{table:dnn-zero-sqrt-mse}
\end{table}

\pagebreak
\subsection{DNN additional experiments}
Before moving to describe the $CNN$ based model, it should be mentioned that several additional experiments were taken based on the $DNN$ model. Two of them are described here.

Trying to help the $DNN$ model to capture the shape of the input, a \textit{sliding-avg} and a \textit{sliding-sum} manipulations on the input have been taken.

\subsubsection{DNN sliding-avg}
By an example, suppose a row of the input is:
\begin{minted}{python}
    [1 2 3 4 5 6 7]
\end{minted}

The \textit{sliding-avg} works row-wise on window sizes of $2^i$ where $i \in \{1,2,...\}$.

In the example above, valid window sizes are:
\begin{minted}{python}
    window_sizes: [2, 4]
\end{minted}

The \textit{sliding-avg} copy a row of the input as is ($1..7$). Then, starts by sliding a window of size 2 (stride=1) and appends the avg. of every two elements ($avg(1,2)$, $avg(2,3)$, ...). Then it begins from the start with a sliding a window of size 4 (stride=1) averaging every 4 elements. The output follows:
\begin{minted}{python}
    [1.  2.  3.  4.  5.  6.  7.  1.5 2.5 3.5 4.5 5.5 6.5 2.5 3.5 4.5 5.5]
\end{minted}

\subsubsection{DNN sliding-sum}
The \textit{sliding-sum} goes the same way as the \textit{sliding-avg} but appends the sum-elements of $window\_size=2$, then $window\_size=4$, etc. The output follows:
\begin{minted}{python}
    [1.  2.  3.  4.  5.  6.  7.  3 5 7 9 11 13 10 14 18 22]
\end{minted}

\subsubsection{Sliding-windows results}
It appeared that the sliding methods didn't improve performance.

\pagebreak
\section{CNN}
Although $DNN$ yields relatively acceptable errors ($\approx 0.01$ for $N>=64$), a $~CNN$ model ($Conv1D$) has been investigated.

A $Conv1D$ layer creates a convolution kernel (filter) that is convolved with its input over a single spatial (or temporal) dimension to produce a tensor of outputs.

Filters learn feature representations of the input. The more representations used the better the performance will be (generally not always, having more filters than needed can also lead to overfitting).

In the first experiment we used a $Conv1D$ with 32-filters of size 3 and a $relu$ activation, followed by max-pooling $(pool\_size=2)$ and two $Dense$ layers $Dense(64, relu)$ and $Dense(1)$. The loss is again $MSE$ and the optimizer used was \textit{Adam}.

The model is fairly simple:
\begin{minted}{python}
def create_cnn_model(n_features, filters=32):
    model = Sequential()
    model.add(Conv1D(32, 3, activation='relu', input_shape=(n_features, 1)))
    model.add(MaxPooling1D(pool_size=2))
    model.add(Flatten())
    model.add(Dense(64, activation='relu'))
    model.add(Dense(1))
    model.compile(loss='mse', optimizer='adam')
    return model
\end{minted}

\subsection{CNN results}
The $CNN$ model was tested the same way as the $DNN$ model using the $trials()$ helper over the same inputs. It yield the following results:

\begin{table}[h!]
    \centering
    \begin{tabular}{||c c c||} 
        \hline
        $N$ & input-shape & $avg(sqrt(mse(\alpha)))$ \\ [0.5ex] 
        \hline\hline
        32 & (10000, 34) & 0.04104 \\ 
        \hline
        64 & (10000, 20) & 0.01080 \\
        \hline
        128 & (10000, 46) & 0.01146 \\
        \hline
        256 & (10000, 21) & 0.00498 \\
        \hline
        512 & (10000, 33) & 0.00365 \\ 
        \hline
        1024 & (10000, 63) & 0.00219 \\ 
        \hline
        2048 & (10000, 140) & 0.00122 \\ 
        \hline
    \end{tabular}
    \caption{$mse(\alpha)$ for $CNN$}
    \label{table:cnn-sqrt-mse}
\end{table}

\subsection{CNN with zero padding}
\par Running \textit{CNN} on the same input as above with right padding zeros yield the following:
\begin{table}[h!]
    \centering
    \begin{tabular}{||c c c||} 
        \hline
        $N$ & input-shape & $avg(sqrt(mse(\alpha)))$ \\ [0.5ex] 
        \hline\hline
        32 & (10000, 44) & 0.04111 \\ 
        \hline
        64 & (10000, 30) & 0.01085 \\
        \hline
        128 & (10000, 56) & 0.01161 \\
        \hline
        256 & (10000, 31) & 0.00483 \\
        \hline
        512 & (10000, 43) & 0.00329 \\ 
        \hline
        1024 & (10000, 73) & 0.00229 \\ 
        \hline
        2048 & (10000, 150) & 0.00110 \\ 
        \hline
    \end{tabular}
    \caption{$mse(\alpha)$ for $CNN$ with zero padding}
    \label{table:cnn-zeros-sqrt-mse}
\end{table}

\section{CNN multi-layer}
The second $CNN$ experiment used a multi-layer with multiple $Conv1D$ and pooling:

\begin{minted}{python}
def create_cnn_model_multi_layer(n_features, filters=32):
    model = Sequential()
    # conv layer 1
    model.add(Conv1D(filters,
                     kernel_size=7, strides=1, 
                     activation='relu', 
                     input_shape=(n_features, 1)))
    # conv layer 2
    input_size = ( input_size + 2 * padding - kernel_size ) / strides  + 1
    model.add(Conv1D(filters, kernel_size=5, ...) 
    # pooling layer 1
    model.add(MaxPooling1D(pool_size=2))
    # conv layer 3
    model.add(Conv1D(filters, kernel_size=3, ...) 
    # pooling layer 2
    model.add(MaxPooling1D(pool_size=2))
    # flatten
    model.add(Flatten())
    # dense layer 1
    model.add(Dense(64, activation='relu'))
    # output layer
    model.add(Dense(1))
    model.compile(loss='mse', optimizer='adam', metrics=['mse'])
    return model
\end{minted}

\newpage
\subsection{CNN multi-layer results}
\par Running \textit{CNN multi-layer} on the same input as above yield the following:
\begin{table}[h!]
    \centering
    \begin{tabular}{||c c c||} 
        \hline
        $N$ & input-shape & $avg(sqrt(mse(\alpha)))$ \\ [0.5ex] 
        \hline\hline
        32 & (10000, 44) & 0.04095 \\ 
        \hline
        64 & (10000, 30) & 0.01097 \\
        \hline
        128 & (10000, 56) & 0.01028 \\
        \hline
        256 & (10000, 31) & 0.00474 \\
        \hline
        512 & (10000, 43) & 0.00302 \\ 
        \hline
        1024 & (10000, 73) & 0.00158 \\ 
        \hline
        2048 & (10000, 150) & 0.00096 \\ 
        \hline
    \end{tabular}
    \caption{$mse(\alpha)$ for $CNN$ multi-layer}
    \label{table:cnn-multi-sqrt-mse}
\end{table}

\subsection{CNN multi-layer with zero padding}
\par Running \textit{CNN multi-layer} with zero padding on the same input as above with right padding zeros yield the following:
\begin{table}[h!]
    \centering
    \begin{tabular}{||c c c||} 
        \hline
        $N$ & input-shape & $avg(sqrt(mse(\alpha)))$ \\ [0.5ex] 
        \hline\hline
        32 & (10000, 44) & 0.04150 \\ 
        \hline
        64 & (10000, 30) & 0.01094 \\
        \hline
        128 & (10000, 56) & 0.01075 \\
        \hline
        256 & (10000, 31) & 0.00438 \\
        \hline
        512 & (10000, 43) & 0.00274 \\ 
        \hline
        1024 & (10000, 73) & 0.00159 \\ 
        \hline
        2048 & (10000, 150) & 0.00068 \\ 
        \hline
    \end{tabular}
    \caption{$mse(\alpha)$ for $CNN$ multi-layer with zero padding}
    \label{table:cnn-multi-zeros-sqrt-mse}
\end{table}

\pagebreak
\section{METHODS COMPARISON}
Table \ref{table:sqrt-mse-comparison} summarise $\sqrt{mse(\alpha)}$ for $N=2^i~ i\in\{5...11\}$ for the different methods used.
\begin{table}[!htb]
    \sffamily
    \scriptsize
    \caption*{
        $z$: right zero padding\\
        $m$: multi-layer CNN
    }
    \centering
    \begin{tabular}{||c c c c c c c||} 
        \hline
        $N$ & $DNN$ & $DNN_z$ & $CNN$ & $CNN_z$ & $CNN_m$ & $CNN_{m,z}$ \\ [0.5ex] 
        \hline\hline
        32 & 0.043260 & 0.043405 & 0.041040 & 0.041108 & 0.040951 & 0.041504 \\ 
        \hline
        64 & 0.014299 & 0.014351 & 0.010801 & 0.010851 & 0.010968 & 0.010939 \\
        \hline
        128 & 0.014803 & 0.015061 & 0.011461 & 0.011613 & 0.010276 & 0.010751 \\
        \hline
        256 & 0.010409 & 0.010662 & 0.004984 & 0.004832 & 0.004736 & 0.004380 \\
        \hline
        512 & 0.009084 & 0.007984 & 0.003653 & 0.003293 & 0.003019 & 0.002745 \\
        \hline
        1024 & 0.009846 & 0.011425 & 0.002195 & 0.002291 & 0.001577 & 0.001586 \\
        \hline
        2048 & 0.010022 & 0.007886 & 0.001215 & 0.001096 & 0.000956 & 0.000675 \\
        \hline
    \end{tabular}
    \caption{$sqrt(mse(\alpha))$ comparison}
    \label{table:sqrt-mse-comparison}
\end{table}

Figure \ref{fig:mse-alpha-comparison} displays a (\textit{logarithmic}) plot of the errors for different $N's$ for each of the methods. The performance of the \textit{CNN} methods over the \textit{DNN} stands out.
\begin{figure}[!htb]
\centering
\includegraphics[width=0.7\linewidth]{./sqrt_mse_compare}
\caption{$mse(\alpha)$ comparison}
\label{fig:mse-alpha-comparison}
\end{figure}

Table \ref{table:STD-comparison} summarise $STD$ of the different methods used.
\begin{table}[!htb]
    \sffamily
    \scriptsize
    \centering
    \begin{tabular}{||c c c c c c c||} 
        \hline
        $N$ & $DNN$ & $DNN_z$ & $CNN$ & $CNN_z$ & $CNN_m$ & $CNN_{m,z}$ \\ [0.5ex] 
        \hline\hline
    	32 & 0.023159 & 0.023944 & 0.025286 & 0.025220 & 0.025482 & 0.024870 \\
    	\hline
    	64 & 0.006112 & 0.005835 & 0.007213 & 0.007136 & 0.007140 & 0.007112 \\
    	\hline
    	128 & 0.007661 & 0.007169 & 0.007652 & 0.007497 & 0.006669 & 0.006585 \\
    	\hline
    	256 & 0.003591 & 0.002905 & 0.003501 & 0.003425 & 0.002951 & 0.003059 \\
    	\hline
    	512 & 0.002553 & 0.002169 & 0.002071 & 0.002125 & 0.001828 & 0.001911 \\
    	\hline
    	1024 & 0.001992 & 0.003733 & 0.001593 & 0.001700 & 0.001009 & 0.001019 \\
    	\hline
    	2048 & 0.003038 & 0.001441 & 0.000609 & 0.000606 & 0.000533 & 0.000425 \\
	\hline
    \end{tabular}
    \caption{$STD$ comparison}
    \label{table:STD-comparison}
\end{table}

\newpage
\subsection{T-TEST}
Is the difference in performance between two machine learning models real, or due to a statistical fluke? Statistical significance tests are designed to address this question.

\textit{T-TEST} compares two averages (means) and tells us if they are different from each other and how significant the differences are; In other words it lets us know if those differences could have happened by chance.

We will use a paired \textit{T-TEST} to compare the results of the different methods used in this project, particularly, the results of \textit{DNN} versus those of \textit{CNN}.

The model\_compare function accepts two models and compares the significance of the difference between their performance (see pseudo code below):

\begin{minted}{python}
from scipy import stats

def model_compare(nn1='DNN', nn2='CNN', N_pow, num_tests)
    N = 2**N_pow
    for i in range(num_tests):
        random_state = random.randint(0,100)
        X, y = generate_data(N, random_state, ...)
        X_train, X_test, y_train, y_test = train_test_split(X, y)
        for nn in [nn1, nn2]:
            model = create_model(nn, X_train, random_state)
            train(model, X_train, y_train, ..., random_state)
            y_pred = model.predict(X_test_model)
            sqrt_mse = np.sqrt( mean_squared_error(y_test, y_pred) )
    a = [ sqrt_mse results of nn1 in each random_state ]
    b = [ sqrt_mse results of nn2 in each random_state ]

    return stats.ttest_rel(a, b)
\end{minted}

Table \ref{table:ttest-comparison} summarize the results of comparing some models.\\
We can see $ttest(DNN, CNN)$ for $N >= 256$ pops out, compatible with Figure \ref{fig:mse-alpha-comparison}.

\begin{table}[h!]
    \scriptsize
    \centering
    \begin{tabular}{||c c c c c||} 
        \hline
        $nn1$ & $nn2$ & $N$ & num\_tests & ttest-$p$ \\ [0.5ex] 
        \hline\hline
        $DNN$ & $DNN_z$ & 64 & 10 & 0.23833 \\ 
        \hline
        $DNN$ & $DNN_z$ & 256 & 5 & \textbf{0.04087} \\ 
        \hline
        $DNN$ & $CNN$ & 64 & 10 & 0.47183 \\ 
        \hline
        $DNN$ & $CNN$ & 128 & 5 & 0.08147 \\ 
        \hline
        \rowcolor{yellow}
        $DNN$ & $CNN$ & 256 & 5 & \textbf{0.01962} \\ 
        \hline
        \rowcolor{yellow}
        $DNN$ & $CNN$ & 1024 & 5 & \textbf{0.00776} \\ 
        \hline
        $CNN$ & $CNN_z$ & 64 & 10 & 0.44374 \\ 
        \hline
        $CNN$ & $CNN_z$ & 256 & 5 & 0.24146 \\ 
        \hline
        $CNN$ & $CNN_m$ & 64 & 10 & 0.89247 \\ 
        \hline
        $CNN$ & $CNN_m$ & 256 & 5 & 0.57363 \\ 
        \hline
    \end{tabular}
    \caption{t-test model comparison}
    \label{table:ttest-comparison}
\end{table}

\section{CONCLUSIONS}

\begin{enumerate}
\small
    \item We have shown (table \ref{table:sqrt-mse-comparison}) that the exponent $\alpha$ of \textit{Yule-Simon} may be predicted using $DNN$ models with an error $\sqrt{MSE(\alpha)} \approx 0.01$ starting from $N \geq 256$.
    
    \item We have also shown that using $CNN$ models, the error is reduced by an order of magnitude ($\sqrt{MSE(\alpha)} \approx 0.001$) starting from $N \geq 1024$.
    
    \item Using \textsc{t-test} (table \ref{table:ttest-comparison}) we have shown that there is a significant difference between the means of the errors produced by the \textit{CNN} models and those produced by the \textit{DNN} models. Thus \textit{CNN} should be preferred.
\end{enumerate}

\section{FURTHER INVESTIGATION}
There are still some open issues left for future investigation.
TODO

\printbibliography
\end{document} 
